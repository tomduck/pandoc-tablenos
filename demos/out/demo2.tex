\begin{longtable}[]{@{}rlcl@{}}
\caption{Demonstration of a simple table. \label{tbl:1}}\tabularnewline
\toprule
Right & Left & Center & Default\tabularnewline
\midrule
\endfirsthead
\toprule
Right & Left & Center & Default\tabularnewline
\midrule
\endhead
12 & 12 & 12 & 12\tabularnewline
123 & 123 & 123 & 123\tabularnewline
1 & 1 & 1 & 1\tabularnewline
\bottomrule
\end{longtable}

% Cleveref macros
\providecommand{\crefformat}[2]{}{}
\providecommand{\Crefformat}[2]{}{}
\crefformat{table}{table~#2#1#3}
\Crefformat{table}{Table~#2#1#3}

% Cleveref fakery
\providecommand{\plusnamesingular}{}
\providecommand{\starnamesingular}{}
\providecommand{\cref}{\plusnamesingular~\ref}
\providecommand{\Cref}{\starnamesingular~\ref}

\renewcommand{\starnamesingular}{Table}\Cref{tbl:1} is from the Pandoc
User's Guide. A simpler table is given by
\renewcommand{\plusnamesingular}{table}\cref{tbl:2}:

\begin{longtable}[]{@{}ll@{}}
\caption{Even simpler. \label{tbl:2}}\tabularnewline
\toprule
A & B\tabularnewline
\midrule
\endfirsthead
\toprule
A & B\tabularnewline
\midrule
\endhead
0 & 1\tabularnewline
\bottomrule
\end{longtable}
